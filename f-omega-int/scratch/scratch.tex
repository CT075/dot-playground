\documentclass[a4paper, 10pt]{article}
\usepackage{amssymb, amsmath, amsthm}
\usepackage{enumitem}
\usepackage{mathrsfs}
\usepackage{parskip}
\usepackage{mathpartir}
\usepackage{stmaryrd}

\newcommand{\Fwint}{\ensuremath{F^\omega_{..}}}
\newcommand{\Dsub}{\ensuremath{D_{<:}}}
\newcommand{\interval}[2]{#1 .. #2}
\newcommand{\isctx}[1]{#1\ \texttt{ctx}}
\newcommand{\iskd}[1]{#1\ \texttt{kd}}
\newcommand{\TyKd}{*}
\newcommand{\KDepArr}[3]{\Pi(#1:#2).#3}
\newcommand{\subst}[3]{#1[#2/#3]}
\newcommand{\isval}[1]{#1\ \texttt{val}}
\newcommand{\isenv}[1]{#1\ \texttt{env}}
\newcommand{\stepsn}[1][]{\Downarrow^{#1}}
\newcommand{\KDenot}[2][\Gamma]{\llbracket #2 \rrbracket_{#1}}
\newcommand{\KEval}[2][\Gamma]{\mathscr{E}\llbracket #2 \rrbracket_{#1}}

\newtheorem{theorem}{Theorem}
\newtheorem{defn}{Definition}
\newtheorem{lemma}{Lemma}

\setlength{\parindent}{0cm}

\title{scratch}
\author{Cameron Wong}

\begin{document}

\section{Heap-based evaluation}

Consider the case of the proof for the rule regarding singleton intervals
below:

\begin{mathpar}
  \inferrule*[right=K-Sing]{\Gamma \vdash A : \interval{B}{C}}
    {\Gamma \vdash A : \interval{A}{A}}
\end{mathpar}

By the inductive hypothesis, we have that there exists some $\alpha, n$ such
that

\begin{mathpar}
  H \vdash A \stepsn[n] \alpha
\end{mathpar}
and
\begin{mathpar}
  \alpha \in \KDenot{\interval{B}{C}}
\end{mathpar}

Our obligation is to show that $\alpha \in \KDenot{\interval{A}{A}}$.

The obvious way to do this is to show that $\Gamma \vdash \alpha \le A$.
However, I don't think this is possible.

Let
\begin{itemize}
  \item $A = X$ (e.g., a variable)
  \item $B, C$ and $\tau$
  \item $\Gamma$ be a context containing $X : \interval{B}{C}$ such that
    $\Gamma \vdash \tau : \interval{B}{C}$
  \item $H$ be a heap containing $X = \tau$
\end{itemize}

Then $H \vdash A \stepsn[0] \tau$, and we know $\tau : \interval{B}{C}$.
However, there's no way to show that $\tau \le X$, because the subtyping rules
never interact directly with variables, except in the reflexivity rule $\Gamma
\vdash X \le X$. Instead, the $\beta$ subtyping rules use substitution.

\subsection{Ideas}

\subsubsection{Heap regularity}

One idea might be to include a premise of $\Gamma \vdash X \le \tau$ in the
definition of consistent environments $\Gamma \ H$. However, this only
defers the problem, as it now becomes impossible to ever construct a proof that
$\Gamma \models H$.

\subsubsection{Amending the subtyping rules}

Another idea would be to adjust the subtyping rules, then show that Stucki's
original formulation is equivalent to ours. However, it's not clear to me what
changes to actually make (or that doing so would be easier than simply fixing
the proof to use substitution rather than a heap in the first place). Changing
the subtyping rules to themselves use a heap would work, but feels very
inelegant.

\section{Totality (RESOLVED)}

My proof uses a denotation function mapping kinds to the set of normalizing
type expressions of that kind as follows:

  \begin{align*}
    \KDenot{\TyKd} &= \{ \langle H, \tau \rangle \} \\
    \KDenot{\interval{A}{B}} &=
      \{ \langle H, \tau \rangle \mid
         \Gamma \vdash A \le \tau : \TyKd,
         \Gamma \vdash \tau \le B : \TyKd,
      \} \\
    \KDenot{\KDepArr{X}{J}{K}} &=
      \{ \langle H, \lambda(X:J).A \rangle \mid
         \forall \tau_x \in \KDenot{J} .
           \langle H(X \mapsto \tau_x), A \rangle \in
           \KEval[\Gamma]{\subst{K}{\tau_x}{X}}
      \} \\
    \KEval{K} &=
      \{ \langle H, A \rangle \mid
         \exists \tau .
         H \vdash A \stepsn \tau \land
         \langle H, \tau \rangle \in \KDenot{K}
    \}
\end{align*}

so strong normalization can be stated as

\begin{mathpar}
  \inferrule*[right=Strong-Normalize]
    {\Gamma \vdash A : K \and \Gamma \models H}
    {\langle H, A \rangle \in \KEval{K}}
\end{mathpar}

which is proven by induction on the judgment $\Gamma \vdash A : K$.

\section{The Issue}

Consider the variable case $A = X$, with the relevant rules being

\begin{mathpar}
  \inferrule*[right=K-Var]{\isctx{\Gamma} \and \Gamma(X)=K}
    {\Gamma \vdash X:K} \and
  \inferrule*[right=Eval-var]{H(X) = \tau}{H \vdash X \stepsn[0] \tau}
\end{mathpar}

To show that $\langle H, X \rangle \in \KDenot{K}$, we need to show $\tau \in
\KDenot{K}$. Even if we assume a strict call-by-value semantics, we only know
that $\Gamma \vdash \tau : K$ and $H \vdash \isval{\tau}$.

One solution would be to bake in $\KDenot{\cdot}$ to the definition of the heap
$H$, but that feels like a mistake.

So we need to prove

\begin{mathpar}
  \inferrule*[right=Denot-Spec]
    {\Gamma \vdash \tau : K \and H \vdash \isval{\tau}}
    {\langle H , \tau \rangle \in \KDenot{K}}
\end{mathpar}

Ideally, this would be trivial, as the entire point of $\KDenot{\cdot}$ is
for this to be true, and it is a mostly straightforward induction on the
judgment $H \vdash \isval{\tau}$. However, we run into an issue in the case
that $\tau = \lambda(X:K).A$, with relevant rules/clauses:

\begin{mathpar}
  \inferrule{\phantom{}}{H \vdash \isval{\lambda(X:K).A}} \\
  \KDenot{\KDepArr{X}{J}{K}} =
    \{ \langle H, \lambda(X:J).A \rangle \mid
       \forall \tau_x \in \KDenot{J} .
         \langle H(X \mapsto \tau_x), A \rangle \in
         \KEval[\Gamma]{\subst{K}{\tau_x}{X}}
    \}
\end{mathpar}

The issue arises when showing that $\langle H(X \mapsto \tau_x), A \rangle \in
\KEval{K}$. Intuitively, this is simple; we just use the main strong
normalization proof. But now we're doing some mutual induction that I'm not
convinced is well-founded -- we invoke \textsc{Strong-Normalize} with $A$
(which is smaller than $\lambda(X:K).A$) and $H,X=\tau_x$ (which is larger than
$H$). But then the proof of \textsc{Strong-Normalize} invokes
\textsc{Denot-Spec} with an \emph{arbitrary} $\tau$ in the var case.

My best guess is that I need to somehow use some measure of the heap size and
term size combined, but I don't know if that actually works -- $\tau$ can be
arbitrarily large, and we don't shrink the heap when we pass it back to
\textsc{Denot-Spec}.

\subsection{Resolution}

Change the definition of $\Gamma \models H$ to include $\tau \in \KEval{K}$.

\end{document}

