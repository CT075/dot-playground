\documentclass[a4paper, 10pt]{article}
\usepackage{amssymb, amsmath, amsthm}
\usepackage{enumitem}
\usepackage{mathrsfs}
\usepackage{parskip}
\usepackage{mathpartir}
\usepackage{stmaryrd}

\newcommand{\Fwint}{\ensuremath{F^\omega_{..}}}
\newcommand{\Dsub}{\ensuremath{D_{<:}}}
\newcommand{\interval}[2]{#1 .. #2}
\newcommand{\isctx}[1]{#1\ \texttt{ctx}}
\newcommand{\iskd}[1]{#1\ \texttt{kd}}
\newcommand{\TyKd}{*}
\newcommand{\KDepArr}[3]{\Pi(#1:#2).#3}
\newcommand{\subst}[3]{#1[#2/#3]}
\newcommand{\isval}[1]{#1\ \texttt{val}}
\newcommand{\isenv}[1]{#1\ \texttt{env}}
\newcommand{\stepsn}[1][]{\Downarrow^{#1}}
\newcommand{\KDenot}[2][\Gamma]{\llbracket #2 \rrbracket_{#1}}
\newcommand{\KEval}[2][\Gamma]{\mathscr{E}\llbracket #2 \rrbracket_{#1}}

\newtheorem{theorem}{Theorem}
\newtheorem{defn}{Definition}
\newtheorem{lemma}{Lemma}

\setlength{\parindent}{0cm}

\title{\Fwint\ Types are Strongly Normalizing}
\author{Cameron Wong}

\begin{document}
\maketitle

\section{Semantics of \Fwint}

The presentation here is taken, slightly modified, from that of Stucki et. al.

We omit the rules concerning terms for brevity.

\begin{figure}[h]
  \begin{align*}
          & X,Y,Z... && \textbf{Type Variable} \\
    A,B,\tau &::= X \mid \top \mid \bot \mid A \rightarrow B \mid
        \forall(X:K).A \mid \lambda(X:K).A \mid A\ B
      && \textbf{Type} \\
    J,K,L &::= \TyKd \mid \interval{A}{B} \mid \KDepArr{X}{J}{K}
      && \textbf{Kind} \\
    \Gamma &::= \varnothing \mid \Gamma, x:A \mid \Gamma, X:K
      && \textbf{Context}
  \end{align*}
  \caption{Syntax of \Fwint}
\end{figure}

\begin{figure}[ht]
  \begin{mathpar}
    \inferrule*[right=C-empty]{\phantom{}}{\isctx{\varnothing}} \and
    \inferrule*[right=C-KdBind]{\isctx{\Gamma}\and\Gamma \vdash \iskd{K}}
      {\isctx{\Gamma, X:K}} \and
    \inferrule*[right=C-TpBind]{\isctx{\Gamma}\and\Gamma \vdash A : \TyKd}
      {\isctx{\Gamma, x:A}}
  \end{mathpar}
  \caption{Context formation}
\end{figure}

\begin{figure}[ht]
  \begin{mathpar}
    \inferrule*[right=Wf-Type]{\phantom{}}{\Gamma \vdash \iskd{\TyKd}} \and
    \inferrule*[right=Wf-Intv]
      {\Gamma \vdash A : \TyKd \and \Gamma \vdash B : \TyKd}
      {\Gamma \vdash \iskd{\interval{A}{B}}} \and
    \inferrule*[right=Wf-DArr]
      {\Gamma \vdash \iskd{J} \and \Gamma, X:J \vdash \iskd{K}}
      {\Gamma \vdash \iskd{\KDepArr{X}{J}{K}}}
  \end{mathpar}
  \caption{Kind formation}
\end{figure}

\begin{figure}[ht]
  \begin{mathpar}
    \inferrule*[right=K-Var]{\isctx{\Gamma} \and \Gamma(X)=K}
      {\Gamma \vdash X:K} \and
    \inferrule*[right=K-Top]{\phantom{}}{\Gamma \vdash \top : \TyKd} \and
    \inferrule*[right=K-Bot]{\phantom{}}{\Gamma \vdash \bot : \TyKd} \and
    \inferrule*[right=K-Arr]
      {\Gamma \vdash A : \TyKd \and \Gamma \vdash B : \TyKd}
      {\Gamma \vdash A \rightarrow B : \TyKd} \and
    \inferrule*[right=K-All]
      {\Gamma \vdash \iskd{K} \and \Gamma, X:K \vdash A : \TyKd}
      {\Gamma \vdash \forall(X:K).A : \TyKd} \and
    \inferrule*[right=K-Abs]
      {\Gamma \vdash \iskd{J} \and \Gamma, X:J \vdash A : K \and
       \Gamma, X:J \vdash \iskd{K}}
      {\Gamma \vdash \lambda(X:J).A : \KDepArr{X}{J}{K}} \and
    \inferrule*[right=K-App]
      {\Gamma \vdash A : \KDepArr{X}{J}{K} \and \Gamma \vdash B : J \and
       \Gamma, X:J \vdash \iskd{K} \and \Gamma \vdash \iskd{\subst{K}{B}{X}}}
      {\Gamma \vdash A\ B : \subst{K}{B}{X}} \and
    \inferrule*[right=K-Sing]{\Gamma \vdash A : \interval{B}{C}}
      {\Gamma \vdash A : \interval{A}{A}} \and
    \inferrule*[right=K-Sub]
      {\Gamma \vdash A:J \and \Gamma \vdash J \le K}{\Gamma \vdash A:K}
  \end{mathpar}
  \caption{Kind assignment}
\end{figure}

\begin{figure}[ht]
  \begin{mathpar}
    \inferrule*[right=SK-Intv]
      {\Gamma \vdash A_2 \le A_1:\TyKd \and \Gamma \vdash B_1 \le B_2:\TyKd}
      {\Gamma \vdash \interval{A_1}{B_1} \le \interval{A_2}{B_2}} \and
    \inferrule*[right=SK-DArr]
      {\Gamma \vdash \iskd{\KDepArr{X}{J_1}{K_1}} \and
       \Gamma \vdash J_2 \le J_1 \and
       \Gamma, X:J_2 \vdash K_1 \le K_2}
      {\Gamma \vdash \KDepArr{X}{J_1}{K_1} \le \KDepArr{X}{J_2}{K_2}}
  \end{mathpar}
  \caption{Subkinding}
\end{figure}

As per Wang and Rompf, we use a runtime environment lookup-based evaluation
strategy rather than a substitution-based one, detailed in figure \ref{rts}.

\begin{figure}[ht]
  \begin{align*}
    H &::= \emptyset \mid H, X=\tau && \textbf{Runtime environment}
  \end{align*}

  \begin{mathpar}
    \inferrule*[right=H-Emp]{\phantom{}}{\isenv{\emptyset}}\and
    \inferrule*[right=H-Bind]{H \vdash \isval{\tau}}{\isenv{H, X=\tau}}
  \end{mathpar}

  \caption{Runtime structures}\label{rts}
\end{figure}

We use the judgment $H \vdash A \stepsn[n] \tau$ to say that $A$ steps to a
type value $\tau$ in $n$ steps. We also use $H \vdash A \stepsn \tau$ as
shorthand for $\exists n . A \stepsn[n] \tau$.

\begin{figure}[ht]
  \begin{mathpar}
    \inferrule{\phantom{}}{H \vdash \isval{\top}} \and
    \inferrule{\phantom{}}{H \vdash \isval{\bot}} \and
    \inferrule{\phantom{}}{H \vdash \isval{\forall(X:K).A}} \and
    \inferrule{\phantom{}}{H \vdash \isval{\lambda(X:K).A}} \and
    \inferrule{H \vdash \isval{A} \and H \vdash \isval{B}}
      {H \vdash \isval{A \rightarrow B}} \and
    \inferrule{H \vdash \isval{A} \and H \vdash \isval{B}}
      {H \vdash \isval{\interval{A}{B}}}
  \end{mathpar}

  \begin{mathpar}
    \inferrule*[right=Eval-var]{H(X) = \tau}{H \vdash X \stepsn[0] \tau} \and
    \inferrule*[right=Eval-arr]
      {H \vdash A \stepsn[n] \tau_A \and H \vdash B \stepsn[m] \tau_B}
      {H \vdash A \rightarrow B \stepsn[n+m] \tau_A \rightarrow \tau_B} \and
    \inferrule*[right=Eval-app]
      {H \vdash A \stepsn[a] \lambda(X:K).A' \and
       H \vdash B \stepsn[b] \tau \and
       H, X=\tau \vdash A' \stepsn[n] \tau'}
      {H \vdash A\ B \stepsn[a+b+n] \tau'} \and
    \inferrule*[right=Eval-intv]
      {H \vdash A \stepsn[a] \tau_A \and
       H \vdash B \stepsn[b] \tau_B}
      {H \vdash \interval{A}{B} \stepsn[a+b] \interval{\tau_A}{\tau_B}}
  \end{mathpar}
  \caption{Type reduction}
\end{figure}

\section{Analysis}

We follow the method of Girard and Tait. For kinds $K$, we define its
denotation $\KDenot{K}$ as the set of type values inhabiting $K$, given some
type environment $\Gamma$. Type variables can appear in kinds in the bounds of
some interval $\interval{A}{B}$, so this context is necessary to enforce that
the correct subtyping relations are satisfied.

\begin{figure}[ht]
  \begin{defn}[Kind-Type Relation]
    \begin{align*}
      \KDenot{\TyKd} &= \{ \langle H, \tau \rangle \} \\
      \KDenot{\interval{A}{B}} &=
        \{ \langle H, \tau \rangle \mid
           \Gamma \vdash A \le \tau : \TyKd,
           \Gamma \vdash \tau \le B : \TyKd,
        \} \\
      \KDenot{\KDepArr{X}{J}{K}} &=
        \{ \langle H, \lambda(X:J).A \rangle \mid
           \forall \tau_x \in \KDenot{J} .
             \langle H(X \mapsto \tau_x), A \rangle \in
             \KEval[\Gamma]{\subst{K}{\tau_x}{X}}
        \} \\
      \KEval{K} &=
        \{ \langle H, A \rangle \mid
           \exists \tau .
             H \vdash A \stepsn \tau \land
             \langle H, \tau \rangle \in \KDenot{K}
        \}
    \end{align*}
  \end{defn}
\end{figure}

\begin{figure}[ht]
  \begin{defn}[Consistent environments]
    \begin{mathpar}
      \inferrule*{\phantom{}}{\varnothing \models \emptyset} \and
      \inferrule*
        {\Gamma \models H \and \Gamma \vdash \tau: K \and \isenv{H,X=\tau}}
        {\Gamma, X:K \models H,X=\tau}
    \end{mathpar}
  \end{defn}
\end{figure}

Unlike Wang and Rompf, we do not need to track bounds in our denotation. In
System \Dsub, interval types are themselves types. This meant that the
\Dsub\ interpretation $\KDenot[\rho]{\text{Type }T_1..T_2}$ could not enforce
any restrictions on its $\text{Type }T$ members while remaining well-founded,
leading to a loss of information when performing semantic widening. In System
\Fwint, however, interval \emph{kinds} are of a different syntactic sort than
the types they restrict, permitting the simpler definition of $\KDenot{\cdot}$.

\subsection{Semantic Subtyping}

\begin{lemma}[Semantic Widening]
  \begin{mathpar}
    \inferrule*
      {\Gamma \vdash K_1 \le K_2}
      {\KDenot{K_1} \subseteq \KDenot{K_2}}
  \end{mathpar}
\end{lemma}

\begin{proof}
  By induction on the subkinding relation.

  \begin{itemize}
    \item Case \textsc{SK-Intv}:

      By transitivity of subtyping.

    \item Case \textsc{SK-DArr}:

      By the inductive hypotheses.
  \end{itemize}
\end{proof}

As bounds can no longer be nested, we do not need to enforce that they are
"good" -- while we can still \emph{construct} the ill-bounded kind
$\interval{\top}{\bot}$, there is no type-level \texttt{null} to provide a
spurious witness.

\subsection{Inversion of Function Kinds}

\begin{lemma}[Inversion of Dependent Functions]\label{depinvert}
  \begin{mathpar}
    \inferrule*
      {\langle H, \tau \rangle \in \KDenot{\KDepArr{X}{J}{K}} \and
       \Gamma \models H}
      {\tau = \lambda (X:J).A \and
       \forall\tau_x \in \KDenot{J} .
         \langle H(X \mapsto \tau_x), A \rangle
         \in \KEval{\subst{K}{\tau_x}{X}}}
  \end{mathpar}
\end{lemma}

\begin{proof}
  By definition of $\KDenot{\cdot}$.
\end{proof}

\subsection{The main proof}

\begin{theorem}[Strong Normalization of types]
  \begin{mathpar}
    \inferrule*{\Gamma \vdash A : K \and \Gamma \models H}
      {\langle H, A \rangle \in \KEval{K}}
  \end{mathpar}
\end{theorem}

\begin{proof}
  By induction on the derivation $\Gamma \vdash A : K$.

  \begin{itemize}
    \item Case \textsc{K-Top}, \textsc{K-Bot}, \textsc{K-All}: Immediate.

    \item Case \textsc{K-Var}: By the definition of consistent environments.

    \item Case \textsc{K-Arr}, \textsc{K-Sing}: By the inductive hypotheses.

    \item Case \textsc{K-Abs}: By the inductive hypothesis.

    \item Case \textsc{K-App}:

      By the inductive hypothesis on $\Gamma \vdash A : \KDepArr{X}{J}{X}$, we
      have $\langle H, A \rangle \in \KEval{\KDepArr{X}{J}{K}}$, so $H \vdash A
      \stepsn[n] \tau$ for some $n$ and $\langle H, \tau \rangle \in
      \KDenot{\KDepArr{X}{J}{K}}$. Then, use lemma \ref{depinvert} and the
      other inductive hypothesis to recover the premises to \textsc{Eval-app}.

      Note that $\tau'$ is a type value and is therefore closed, so it is valid
      to strengthen $\Gamma, X: \tau_B$ back to $\Gamma$

    \item Case \textsc{K-Sub}: By the inductive hypothesis and semantic
      widening.
  \end{itemize}
\end{proof}

\appendix

\section{Structural Lemmas}

\begin{lemma}[Substitution commutes with normalization under denotation]
  \begin{mathpar}
    \inferrule*
      {\Gamma \vdash \iskd{K} \and
       X \not\in \Gamma \and
       \Gamma \models H \and
       H \vdash A \stepsn[n] \tau}
      {\KDenot{\subst{K}{A}{X}} = \KDenot{\subst{K}{\tau_x}{X}}}
  \end{mathpar}
\end{lemma}

\begin{proof}
  By induction on the judgment $H \vdash A \stepsn \tau$.
\end{proof}

\begin{lemma}[Big-step evaluation results in a value]
  \begin{mathpar}
    \inferrule*{\Gamma \models H \and H \vdash A \stepsn[n] \tau}
      {\Gamma \vdash \isval{\tau}}
  \end{mathpar}
\end{lemma}

\begin{proof}
  By induction on the stepping judgment.
\end{proof}

\begin{lemma}[Type values are closed]
  \begin{mathpar}
    \inferrule*{H \vdash \isval{\tau}}{\emptyset \vdash \isval{\tau}}
  \end{mathpar}
\end{lemma}

\begin{proof}
  By induction on the judgment $H \vdash \isval{\tau}$.
\end{proof}

\begin{lemma}[Weakening/Strengthening]
  The type-kind relation is invariant under extending and shrinking the
  context.

  \begin{mathpar}
    \inferrule*{X \not\in FV(K)}{\KDenot{K} = \KDenot[\Gamma(X \mapsto K')]{K}}
  \end{mathpar}
\end{lemma}

\begin{proof}
  By induction on $K$.
\end{proof}

\end{document}

